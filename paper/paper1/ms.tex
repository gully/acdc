\documentclass[modern]{aastex631}
\bibliographystyle{aasjournal}

% \usepackage{fontspec}
% \usepackage[T1]{fontenc}
% \usepackage{newtxsf}
% \setmainfont{Fira Sans Book}[Scale=1.0]

\usepackage[caption=false]{subfig}
\usepackage{censor}
\usepackage{booktabs}

\begin{document}
\shorttitle{Constraints on Starspot Contrast with HPF}
\shortauthors{TBD}
\title{Physical properties of starspots in Nearby Young Moving Groups}

\author{TBD}
\affiliation{University of Texas at Austin Department of Astronomy}

\author{TBD}
\affiliation{TBD}


\begin{abstract}

  Abstract goes here.

\end{abstract}

\keywords{High resolution spectroscopy (2096)}

\section{Introduction}\label{sec:intro}

Here is an annotated bibliography.

Here is some more text! Here is $\vec y=a \vec x^2+b \vec x + c$

\begin{deluxetable}{chc}
  \tablecaption{Annotated bibliography for intro\label{table1}}
  \tablehead{
    \colhead{Reference} & \nocolhead{two} & \colhead{Key idea}
  }
  \startdata
  \citet{gullysantiago17} & - & LkCa~4 starspot spectral decomposition\\
  \citet{2015ApJ...807..174S} & - & Radius inflation from spots \\
  Strassmeier & - & Starspots review \\
  Rackham & - & TLSE \\
  \enddata
\end{deluxetable}

\section{Observations and Data Reduction}
\subsection{TESS}
We programmatically acquired lightcurves from the Transiting Exoplanet Survey Satellite \citep[TESS;][]{2014SPIE.9143E..20R} via the \texttt{lightkurve} Python library \citep{2018ascl.soft12013L}.  All \censorbox{7?} sources had at least one sector of TESS data available in the Full Frame Image (FFI) data.  Targets UCAC2~7201471 and \censorbox{other targets...} also had 2 minute cadence data available.  For these 2-minute cadence data we retrieve the Science Processing Operations Center (SPOC) lightcurves which have already undergone data quality assurance checks \citep{2020RNAAS...4..201C}.  For the FFI data, such quality assurance checks are not automatically applied, so we visually inspect the raw lightcurves, and apply standard masking for quality flags and obvious instrumental artifacts.  These lightcurves show such conspicuous signal strengths, that advanced lightcurve processing was not needed, and the raw flux time series can be used with minimal postprocessing.  Figure \ref{TESS_UCAC2} shows the TESS lightcurve for the target UCAC2~7201471.

\clearpage

\begin{rotatetable}
  \pagestyle{empty}
  \begin{deluxetable*}{lccCCCcCccc}
    \centering
    \tablecaption{Properties of the sample\label{table2}}
    \tabletypesize{\scriptsize}
    \setlength{\tabcolsep}{0.04in}
    \tablewidth{0pt}
    %\rotate
    \tablehead{
      \colhead{Source} &
      \colhead{Sp.T.\tablenotemark{a}} &
      \colhead{Membership} &
      \colhead{Distance} &
      \colhead{$T_\mathrm{eff}$\tablenotemark{b}} &
      \colhead{$\log{g}$} &
      \colhead{$[\mathrm{Fe}/\mathrm{H}]$} &
      \colhead{$RV$\tablenotemark{b}} &
      \colhead{$L_\star$} &
      \colhead{$N_\mathrm{IGRINS}$} &
      \colhead{References}\\
      \colhead{} &
      \colhead{} &
      \colhead{} &
      \colhead{(pc)} &
      \colhead{(K)} &
      \colhead{} &
      \colhead{} &
      \colhead{km/s} &
      \colhead{$L_\odot$} &
      \colhead{(\#)} &
      \colhead{}
    }
    \startdata
    UCAC2 7201471 & M2 & COL & 85.3\pm0.1 & 3511 & 4.2 & -0.37 & x         & - & 6 & GF18, GDR3\\
    V* V1249 Cen (TWA 25) & M0.5 & TWA & 53.6\pm0.1 & -    & -   & -     & 6.1\pm1.6 & - & 2 & SZB03, HH14, GDR3\\
    HD 49855 (HIP 32235) & G6V & CAR & 55.31\pm0.05 & 5399\pm4 & 4.45 & -0.07 & 20.658 & - & 5 & T06, GDR3 \\
    TYC 8534-1243-1 & - & - & - & - & - & - & - & - & 5 & T06, GDR3 \\
    V* V479 Car & - & - & - & - & - & - & - & - & 5 & GDR3 \\
    %2MASS J12271665-6239142 & - & - & - & - & - & - & - & - & 1 & GDR3 \\
    %CD-62 657 & - & - & - & - & - & - & - & - & 1 & GDR3 \\
    \enddata
  \end{deluxetable*}
\end{rotatetable}
\tablecomments{This table lists physical properties for the sample as compiles from literature sources.}
\tablenotetext{a}{Spectral Type classification, usually in the visible wavelengths}
\tablenotetext{b}{Coarse estimates, typically from Gaia DR3}
\tablerefs{(GF18) \citet{2018ApJ...862..138G}; (GDR3) \citet{2016A&A...595A...1G} Gaia Collaboration, A. Vallenari, et al. (2022j); (SZB03) \citet{2003ApJ...599..342S}, (HH14) \citet{herczeg14}, (T06) \citet{2006A&A...460..695T}}


\begin{figure}[htb]
  \epsscale{1.0}
  \plotone{figures/UCAC2_7201471_TESS_visits.png}
  \caption{\label{TESS_UCAC2} TESS Sector 34 and 35 lightcurves for UCAC2~7201471.  The two sectors of data are separately normalized by their median values.  The six vertical blue lines indicate the moments of acquisition of IGRINS spectra from coordinated Gemini South observations.  The final IGRINS observation occurred just after TESS Sector 35 concluded. The dotted horizontal red line at flux of 0.905 highlights the lower envelope of flux minima for this source, indicating a peak-to-valley flux loss of 9.5\%.  This large lightcurve modulation suggests a significant longitudinal asymmetry of starspot coverage fraction with at least a 10\% coverage fraction of spots on the most spotted hemisphere.}
\end{figure}

\subsection{IGRINS}
IGRINS acquired \censorbox{35?} individual spectra of 7 objects in Gemini South semester 2021A under program ID \texttt{GS-2021A-Q-311}.


\section{Results}

\begin{deluxetable}{lcccc}
  \centering
  \tablecaption{Derived properties\label{table3}}
  \tablehead{
    \colhead{} &
    \colhead{TESS} &
    \colhead{TESS} &
    \colhead{ASASSN $V$} &
    \colhead{ASASSN $g$} \\
    \colhead{Source} &
    \colhead{Period} &
    \colhead{Amplitude} &
    \colhead{Amplitude} &
    \colhead{Amplitude} \\
    \colhead{} &
    \colhead{(d)} &
    \colhead{(\%)} &
    \colhead{(\%)} &
    \colhead{(\%)}
  }
  \startdata
  UCAC2 7201471 & - & - & - & - \\
  V* V1249 Cen & - & - & - & -\\
  HD 49855 & - & - & - & -\\
  TYC 8534-1243-1 & - & - & - & -\\
  V* V479 Car & - & - & - & -\\
  \enddata
\end{deluxetable}
\tablecomments{Amplitudes are defined as flux losses from the normalized peak.}

\begin{acknowledgements}
  This paper includes data collected by the TESS mission. Funding for the TESS mission is provided by the NASA's Science Mission Directorate.

  This research made use of Lightkurve, a Python package for Kepler and TESS data analysis \citep{2018ascl.soft12013L}.

  The authors acknowledge the Texas Advanced Computing Center (TACC, \url{http://www.tacc.utexas.edu}) at The University of Texas at Austin for providing HPC resources that have contributed to the research results reported within this paper.

  This work has made use of data from the European Space Agency (ESA) mission {\it Gaia} (\url{https://www.cosmos.esa.int/gaia}), processed by the {\it Gaia} Data Processing and Analysis Consortium (DPAC,\url{https://www.cosmos.esa.int/web/gaia/dpac/consortium}). Funding for the DPAC
  has been provided by national institutions, in particular the institutions
  participating in the {\it Gaia} Multilateral Agreement.

\end{acknowledgements}

\clearpage


\facilities{Gemini:South, TESS, ASAS, Gaia}

\software{  pandas \citep{mckinney10, reback2020pandas},
  emcee \citep{foreman13},
  matplotlib \citep{hunter07},
  astroplan \citep{astroplan2018},
  astropy \citep{exoplanet:astropy13,exoplanet:astropy18},
  exoplanet \citep{exoplanet:exoplanet},
  numpy \citep{harris2020array},
  scipy \citep{jones01},
  ipython \citep{perez07},
  starfish \citep{czekala15},
  bokeh \citep{bokehcite},
  seaborn \citep{waskom14}}
%pytorch \citep{NEURIPS2019_9015}} % No pytorch yet!


\bibliography{ms}


\clearpage

\appendix
\restartappendixnumbering

\section{Optional appendix} \label{appendix:tools}

Place optional content here.
\end{document}
